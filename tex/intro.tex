\clearemptydoublepage
\chapter{Preface to the Twelfth Edition}




The twelfth edition contains approximately 300 additions, of which
100 are newly listed agents.  A special effort has been made to
obtain as much information as possible on drugs and other agents to
which pregnant women may be exposed. Because of an increased
awareness of the importance of negative published data, a
considerable number of nonteratogenic agents have been included.
Many such references are drawn from the Japanese literature, and the
authors are again grateful for the help of Dr. Takashi Tanimura,
Professor of Anatomy at Kinki University School of Medicine. 
Dr.~Takahide Kihara of Kinki University also
helped with the Japanese literature.

Some of the advantages of using a computer program for storage and
production have been realized again with this edition and in many other
modern texts.  The
originally published material theoretically does not need to be
proofread; the names, references, and dosages cannot be mistakenly
altered.  When the program is completed, the entire final printout to
be sent to the publisher can be run off within hours.  Once in the
hands of the publisher, the material does not need to be typeset, a
consideration that reduces the cost of the book. Finally, the
database can be used by agencies such as the National Library of
Medicine to extract specific data and references on various agents
and supply their many clients rapidly.  An  update of this
database is maintained at the University of Washington.  Quarterly
updates are available on CD-ROM through Micromedex, Inc., Greenwood
Village, Colorado.

There are several other sources of teratology information.  James G.
Wilson's text (1973) remains one of the best introductions to
teratology.  A comprehensive four-volume set of books on methods and
principles in teratology was edited by J. G. Wilson and F. C.
Fraser (1977).  Schardein (2000) revised his \textit{Chemically
Induced Birth Defects}, a volume containing a large number of
references and excellent reviews of different agents.  
Schardein and Macina (2007) have published a useful text
on human developmental toxicants.
Heinonen et
al.~(1977) summarize drug exposures during more than 50,000
pregnancies and furnish useful tables of malformation risk factors,
nearly all of which are not significantly increased.  The National
Library of Medicine maintains a Developmental and Reproductive
Toxicology (DART) database.  The older references (before 1989) are
available on ETICBACK, whose system of computerized information
retrieval on teratology is available for those who require an
in-depth listing of references.  Their listing can be obtained
directly through the National Library of Medicine (TOXNET).

Five especially useful publications are available on methods for detection of
teratogenic agents and protection of the human population.
The proceedings of a conference in Guadeloupe under the
auspices of the Institute de la Vie address in vitro, pregnant animal
testing, and epidemiologic methods (Shepard et al., 1975).  Brent and
Harris (1976) edited a volume emphasizing the scientific and
epidemiologic techniques useful to prevent fetal and neonatal
loss. A volume from a conference supported by the March of Dimes
Birth Defects Foundation gives detailed guidelines for studies of
human populations (Bloom, 1981). The U.S. Environmental Protection
Agency published the proceedings of two conferences on techniques for
risk assessment of substances that affect reproduction. Discussions
of protocols for testing the effect of environmental substances on
the male and female reproductive systems and the conceptus are given
(Galbraith et al., 1982). Barlow and Sullivan (1982) summarized data
on reproductive hazards in industry and this contains valuable
information on a number of chemicals.

In the past few years there has been an increase in the demand for
information concerning reproductive risk, and a number of delivery
systems have been developed or expanded. TERIS is a database on
teratologic information and risks and is available via modem, on hard
disk, or on CD-ROM. This information has been reviewed by six
nationally recognized clinical teratologists (Friedman et al., 1990)
and is linked to an updated version of the
{\it Catalog of Teratogenic Agents}. Anthony Scialli
maintains another online database, REPROTOX (Scialli, 1990),
and a published version has appeared (Scialli et al., 1995).
A text by Briggs et al.~(1998),
entitled {\it Drugs in Pregnancy and Lactation},
gives clinical data on many drugs.
The TERIS information has been revised recently (Friedman and Polifka, 2000).
Online versions of TERIS, REPROTOX, and Shepard's Catalog are
available in most medical libraries 
(Reproductive Module of Micromedex, 
Greenwood Village, CO).


Beginning with the eighth edition, this book has been formatted with \LaTeX\ 
at the University of Washington. This has allowed for
much improved clarity and the use of Greek symbols,
boldface type, and special
punctuation. Entries containing a great deal of data have been
reorganized using
boldface headings. The section on occupational
exposures is 
alphabetical by type of work.
A number of interesting birth defect syndromes
have been created by transgenic insertions;
to list these, the organ affected has been
used (for example, an insertion that interferes with
pancreatic development is listed
under pancreatic ablation).

Dr. Shepard is very pleased again to have Dr.~Ronald Lemire as a co-author.
Dr.~Lemire was Shepard's first postdoctoral research fellow in
teratology and he is a professor of pediatrics at the University 
of Washington.  He is an expert in defects of the central nervous 
system and has an active teaching program at the Children's Hospital
and Regional Medical Center of Seattle.

\begin{references}

\item Barlow, S. M., and Sullivan, F. M. {\it Reproductive Hazards of
Industrial Chemicals}. London: Academic Press, 1982.
\item Bloom, A.~D., ed. {\it Guidelines for Studies of Human Populations
Exposed to Mutagenic and Reproductive Hazards}. White Plains, N.Y.:
March of Dimes Birth Defects Foundation, 1981.
\item Brent, R.~L., and Harris, M., eds. {\it Prevention of Embryonic,
Fetal, and Perinatal Disease}. Bethesda,
Md.: DHEW (NIH) 76--853, 1976.
\item Briggs, G.~G., Freeman, R.~K., and Yaffe, S.~J. {\it Drugs in
Pregnancy and Lactation}, 7th ed. Baltimore:
Williams and Wilkins, 2005.
\item Friedman, J.~M., and Polifka, J.~E. {\it Teratogenic Effects of
Drugs: A Resource for Clinicians} (TERIS), 2nd ed. Baltimore:
Johns Hopkins University Press, 2000.
\item Friedman, J.~M., Little, B.~B., Brent, R.~L., Cordero, J.~F., Hanson,
J.~W., and Shepard, T.~H. Potential
human teratogenicity of frequently prescribed drugs.
{\it Obstet. Gynecol}. 75: 594--99, 1990.
\item Galbraith, W.~M., Voytek, P., and Ryan, M.~G., eds.
{\it Assessment of Risks to Human Reproduction
and to Development of the Human Conceptus from Exposure to
Environmental Substances}.
EPA-600/9-82-001.
Springfield, Va.: National Technical Information Service, 1982.
\item Heinonen, O.~P., Slone, D., and Shapiro, S. {\it Birth Defects and
Drugs in Pregnancy}. Boston:
John Wright PSG, Inc., 1977.
\item Schardein, J.~L. {\it Chemically Induced Birth Defects}, 3rd ed.
New York:
Marcel Dekker, 2000.
\item Schardein, J.~L., and Macina, O.~T. {\it Human Developmental Toxicants:
Aspects of Toxicology and Chemistry}. 
Boca Raton, Fla.: CRC Press, Taylor \& Francis, 2007. 

\item Scialli, A.~R. Making information work and Reprotox agent
list additions. {\it Reprod. Toxicol}. 4:249--50,
343--44, 1990.
\item Scialli, A.~R., Lione, A., and Boyle, G.~K.~B.
{\it Reproductive Effects of Chemical, Physical, and Biologic Agents,
Reprotox$^{\hbox{\tiny R}}$.} Baltimore: Johns Hopkins University Press, 1995.
\item Shepard, T.~H., Miller, J.~R., and Marois, M., eds.
{\it Methods for Detection of Environmental Agents
That Produce Congenital Defects}. New York:
North Holland-American Elsevier, 1975.
\item Wilson, J.~G. {\it Environment and Birth Defects}. New York:
Academic Press, 1973.
\item Wilson, J.~G., and Fraser, F.~C., eds. {\it Handbook of Teratology},
vols. 1--4. New York: Plenum Press,
 1977.
\end{references}


\clearemptydoublepage
\chapter{Acknowledgments}

We wish to thank the many unnamed colleagues who have supplied us
with the information included in this catalog. Dr.~Shepard's students
and research fellows have contributed considerably, as have other
teratologists. Our associates Drs. Alan Fantel, Philip Mirkes, and
Janine Polifka were particularly helpful. Dr.~Takashi Tanimura and 
Dr.~Tokahidi Kihara of
the Department of Anatomy at Kinki University School of Medicine
(Osaka) aided with the selection and translation of the Japanese
literature.

The computerization of the book was simplified initially by the
generous help of Dr. Victor A.~McKusick, who allowed us to make use
of a program similar to that employed for production of his catalog
({\it Mendelian Inheritance in Man} [8th ed.; Johns Hopkins
University Press, 1988]). Dr. Richard Shepard and David Bolling of
the Johns Hopkins University School of Medicine and James E.
Peoples of the University of Arizona School of Medicine kindly
helped to provide the edit-and-print programs for computerizing the
body of the first edition of this book.

The computerization of the material and the processing of the catalog
took place in the computer center of the University of Washington
under the able direction of Jim Fox.  Tom Stebbins designed the
figures in the endpaper tables.

Patricia Stark entered our handwritten additions into the computer
database.  Dr. Janine E. Polifka of the Department of Pediatrics,
University of Washington, aided us by identifying typographical and
other errors. Alan Fantel has provided us with a continuous source of information.

The financial assistance of the National Institutes of Health (HD 00836)
is appreciated.


\clearemptydoublepage
\chapter{Introduction}


Defects existing at birth, irrespective of their cause, create
societal problems of such magnitude that the importance of the 
subject needs little
amplification. Approximately 3 percent of all human newborns have a
congenital anomaly requiring medical attention, and approximately
one-third of these conditions can be regarded as life threatening.
With increasing age, more than twice as many congenital defects are
detected.  Close to 40 percent of hospitalized children are there because
of prenatally acquired malformations of one kind or another.

Our knowledge about the cause and prevention of these problems is
extremely limited. About 25 percent are associated with gene
mutations and another 5 percent with chromosomal aberrations. About
10 percent of the remaining anomalies are known to be due to a
teratogenic agent.  There are more than 3,300 agents listed in this
catalog. About 1,500 can produce congenital anomalies in experimental
animals, but only about 40 of these are known to cause defects in 
humans. Therefore, there exists a wide gap between our
knowledge of experimental teratology and our knowledge of
the role that external
agents play in producing human malformations.

A further problem is that the teratologic literature can be found in
most of the biomedical publications rather than being
confined to one or two of the specialized journals.  In addition to
the publications listed in the preface there are a number of related ones
that are useful.  Harold Kalter's (1968) text deals with the
teratology of the central nervous system in animals, and Josef
Warkany's (1971) extensive treatise is concerned with congenital
defects in humans. Jones's (2006) popular book on dysmorphic
syndromes helps to define and extend our description of human
defects.  McKusick's (1998) catalog of Mendelian inheritance in man
includes many congenital syndromes produced by gene mutations;  an
online version of this book is available (Pearson et al., 1994).
Bishop et al.~(1997) give a detailed summary of the genotoxic
aspects of human teratogens.

One of the main purposes of this catalog is to help link the information on
experimental teratogenic agents to congenital defects in humans.
The catalog provides a reference for
teratologists on the literature dealing with a
particular teratogenic agent. For the obstetrician, pediatrician, and
geneticist it should help answer the often-asked question: Does this
agent produce congenital defects in humans or animals? Another
function of this book may be to aid the scientists who protect us
from our modified environment. Testing pharmaceutical products has
become a major responsibility, and the environmental protectionist
must also address the safety of food additives and household products.
This catalog may be useful to chemists working to develop new
products. Unfortunately, because of species variability, at least in
part, in teratogenic sensitivity, the ultimate testing of some
products has been done in the human, with the alert clinician acting
as monitor.

It is important that the presently available information on
teratogens be fully utilized to recognize potential teratogenic
hazards and to prevent congenital malformations in humans. If
many anomalies will be produced by the interaction of genes
with a teratogenic agent, it undoubtedly will be easier to remove the
agent than to alter the action of the gene. The work of producing
this book will be rewarded if it contributes to the prevention of any
congenital defect.

The teratogenic agents listed in this book include chemicals, drugs,
physical factors, gene mutations, and viruses.  We have attempted to
make a comprehensive listing of all agents that have been studied for
teratogenicity in animals or humans. The chemical names are, in most
cases, those that appear in the {\it Merck Index}, but cross-indexing
to alternate chemical and proprietary names has also been done.
Generally, studies carried out on species phylogenetically below the chick have
been omitted. No attempt has been made to list agents that are
teratogenic only when administered in combination with another agent.
The presence of an agent in the catalog does not necessarily indicate
that it is a teratogen, because a number of compounds with
substantially negative teratogenic effects have been included.  When
there is conflicting evidence of teratogenicity, equal representation
of the evidence has been attempted
(an example of this is cigarette smoking).

With the eleventh edition and the present edition we have added over
350 developmental
genes that when altered produce defects in animals or humans.  If a
human syndrome is associated, the name of the gene is cross-indexed.  This
beginning in a rapidly expanding area of investigation was
substantially aided by our associate Dr. Philip Mirkes.

The literature has been surveyed using the usual library aids. 
Abstracts have been avoided where possible. Numerous
excellent references dealing only with the mechanism of a defect's
production have been omitted. Although some agents may have been
omitted inadvertently from this listing, the method of production and
printing of this type of book readily allows for easy revision.  We
hope that scientists in the field will feel free to send us new
information or corrections for inclusion in future editions.

Each listing includes a main entry with synonyms.  In the twelfth
edition, an attempt has been made to list any human data in the first
part of the abstract. This is followed by a brief account of some of
the work published, including species, dose, gestational age at time
of administration, and type of congenital defects produced. The
references following each entry were chosen because of their review
nature, originality, or because they are the most current.

Because a teratogenic agent is defined by its ability to produce a
congenital defect, it seems appropriate to provide a definition for
{\it congenital defect} at this point. A congenital defect has its
genesis during embryonic or fetal development and is a major
or minor deviation from normal morphology or function. Distinguishing
between a minor congenital defect and normal variation is most
difficult, and this accounts for a large difference in
incidence rates.  Jones (2006) offers some criteria for
distinguishing the two.  In general, a minor defect would be present
in less than 2--3 percent of the population. This small
percentage could be defined in a statistical manner as the number of
observations falling outside of three standard deviations from the
mean. Large morphological types of congenital defects, such as cleft
palate or meningomyelocele, may be called anomalies or malformations,
but we feel the term \textit{defect} also includes unknown or subtle
structural defects that alter function. These functional changes also
could be ascribed to molecular changes, many of which are still
unknown. Particularly in the nervous and endocrine systems, changes
in postnatal function have become an important aspect of both
experimental and human teratology. An example of this is maternal
hyperphenylalani\-nemia, which may lead to cerebral dysfunction and
mental retardation in offspring. Another new and important
teratologic area requiring long-term postnatal observation is
prenatally induced oncogenesis, illustrated by vaginal carcinomas
produced in grown girls whose mothers were treated with
diethylstilbesterol during pregnancy.

\enlargethispage*{2pc}
A teratogenic agent acts during pregnancy to produce a physical or
functional defect in the conceptus or offspring. The definition can
be made more specific by using a modification of Koch's postulates in
the following manner:
\pagebreak

\begingroup
\newdimen \leftcolwidth\leftcolwidth=\hsize
  \divide\leftcolwidth by5
  \multiply\leftcolwidth by2
\newdimen \rightcolwidth \rightcolwidth=\hsize
  \advance\rightcolwidth by-\leftcolwidth
  \advance\rightcolwidth by-20pt
\def\kline#1#2{\hbox to\hsize{\vtop{\hsize\leftcolwidth#1\strut}\hss
    \vtop{\hsize\rightcolwidth#2\strut}}}
\def\kcol#1#2{\noindent\hangindent1pc\hangafter1\hbox to1pc{#1\hss}#2}

\parindent0pt
%  \vspace*{12pt}

\kline{\textit{Koch's Postulates}}{\textit{Application to Teratology}}
\vspace*{6pt}
\kline{\kcol{1.}{A specific microorganism must be present in each case.}}
 {\kcol{1.}{The agent must be present during the
 critical periods of development.}}
\vspace*{2pt}
\kline{\kcol{2.}{A pure culture of the organism should
  produce a similar disease in the experimental animal.}}
{\kcol{2.}{The agent should produce congenital
defects in an experimental animal.
The defect rate should be statistically
higher in the treated group than in the
control animals receiving the same
vehicle or sham procedure.}}
\vspace*{2pt}
\kline{\kcol{3.}{Organisms from the experimental
animal must be recovered and grown
in pure culture.}}
{\kcol{3.}{Proof should be obtained that the
agent in an unaltered state acts on
the embryo-fetus either directly or indirectly through the placenta. In this
area, biochemistry and in vitro culture are most often used instead of
  bacteriology.}}

\endgroup
\vspace{12pt}

Fulfillment of the first two conditions is sufficient to define a
teratogenic agent. The third is desirable but not
essential. In this catalog, there are few teratogens that fit all
three of these criteria. Surprisingly, teratogenic agents in the
human generally do fill all three criteria---for instance, rubella
virus, radiation, and androgens that masculinize the female fetus.
Thalidomide, although accepted universally as a teratogen, does not
fit the third criterion because the compound in its unaltered state
has not been demonstrated to directly affect the conceptus. Although
this third criterion may seem unnecessary to many teratologists, we
believe that a more complete knowledge of these important molecular
mechanisms can generate more rapidly the means for preventing
malformations.  An expanded discussion of the criteria for human
teratogenicity can be found on page \pageref{cfht}.

Several difficult problems face the teratologist in judging whether
an agent is a human teratogen. When the dosage level of a compound
must be raised to near-fatal levels for the mother before defects are
produced in her fetus, most workers consider the agent weakly
teratogenic. However, clinicians, the Food and Drug Administration,
and the pharmaceutical industry encounter difficult decisions in
applying these experimental findings to humans. Agents causing
embryonic or fetal death in experimental animals often later prove to
be teratogenic in humans but are not considered to be teratogens
unless physical or functional defects are produced. Similarly, an
agent that can cause fetal growth retardation does not necessarily
qualify as a teratogen. Retardation of fetal skeletal maturation
reported as decreased ossification centers of the manubrium or
immaturity of the vertebral centra is another example of a change
considered physiologic but not teratogenic.

The concept of a ``termination period'' can be helpful in deciding
whether an agent might be responsible for causing a malformation
(Warkany, 1971).  The termination period is the time beyond which a
malformation cannot occur.  For example, the caudal neuropore closes
at 27 days (Carnegie stage 12).  A drug administered on the 35th day
would not be a factor in the causation of a meningomyelocele.

\vspace{2pc}

\begin{references}

\item Bishop, J.~B., Witt, K.~L., and Sloane, R.~A.
Genetic toxicities of human teratogens. {\it Mutat. Res.}
 396:9--43, 1997.
\item Jones, K.~L. {\it Smith's 
Recognizable Patterns of Human Development}, 6th ed. Philadelphia:
Elsevier \& Saunders, 2006.
\item Kalter, H.
{\it Teratology of the Central Nervous System}.
Chicago: University of Chicago Press, 1968.
\item McKusick, V.~A. {\it Mendelian Inheritance in Man:
A Catalog of Human Genes and Genetic Disorders}, 
12th ed. Baltimore:
Johns Hopkins University Press, 1998.
\item Pearson, P., Francomano, C., Foster, P.,
 Bocchini, C., Li, P., and McKusick, V.
 The status of Online Mendelian Inheritance in Man (OMIM).
 {\it Nucleic Acids Res.} 22:3470--73, 1994.
\item Warkany, J. {\it Congenital Malformations: Notes and Comments}.
Chicago: Year Book Medical Publishers, 1971.
\end{references}


\clearemptydoublepage
\chapter{Notes for Table on Endpapers}

\vspace*{-1pc}
{\sffamily \Large
Comparative Time Periods of Embryonic and Fetal Development\\
in Humans and Experimental Animals
}

\vspace{24pt}



\noindent
One of the main principles of teratology is that teratogenic activity
is strongly dependent on exposure to an agent at a specific
``sensitive'' period in development. It seemed appropriate,
therefore, to attempt an integration of embryonic and fetal events
for the human and some common experimental animals. We have tabulated
events that occur in the human, rhesus monkey, rat, mouse, rabbit,
hamster, guinea pig, and chick. The times are given in days from
copulation or ovulation for the monkey, mouse, rabbit, hamster, and
guinea pig. It is assumed that ovulation time is
approximately the same as copulation time. For humans, the time is
derived by subtracting 14 days from the period between the onset of
the last normal menses and time of delivery. For the rat, timing
customarily has been from 9:00 a.m. following copulation, and we have
chosen to follow this method. Timing in the chick is from the start
of incubation.

The reader may note that the ages given for humans are 2--3 days
greater than those originally given by G. L. Streeter, who based
his timing on comparable stages in the monkey. This adjustment, based
on last menstrual period dates, has been drawn largely from material
collected in the Central Laboratory for Human Embryology at the
University of Washington (Iffy et al., 1967; Shepard, 1975; 
Shepard et al., 1988).  Our data, based on North American and
European specimens, agree quite well with the Japanese observations
of Nishimura and Yamamura (1969).  Exact dating of the early somite
stages is still hazardous because of the meager number of available
human embryos.

A good deal of confusion exists about how to refer to 
the days of the pregnancy.
In this volume, and, we believe, as a general
practice among teratologists, the first 24 hours after fertilization
is designated as the first day; however, if the word {\it day}
precedes the cardinal number, this indicates that the age is at least
that many days.  Thus, a day-10 embryo is at least 10 days of age,
but an embryo of the 10th day is between 9 and 10 days of age.
Kalter (1968) eloquently discusses this problem.

Many definitions of the embryonic period have been offered. We define
the end of the human embryonic period or start of the fetal period by
these criteria:  54--60 gestational days, crown--rump length of 33 mm,
end of Carnegie stage 23, and end of major organogenesis. Two other
criteria that might help define the difference in some of the
experimental animal models are the appearance of ossification and the
presence of external characteristics specific for the species.

Data on fetal growth, though less precise, is available.
Tanimura et al.~(1971) designated fetal organ weights for
the human; and Kerr et al.~(1969), for the monkey. Gruenwald and Minh
(1960), Potter (1961), and Shepard et al.~(1988) gave organ weight
standards for fetuses weighing more than 500 gm.

A comprehensive review of naturally occurring malformations in rats,
rabbits, and mice has been published (Nakatsuka et al., 1997).

The endpaper references that follow are those that 
were useful in constructing the tables
appearing on the front and back inside covers of the book.

\newpage

\begin{references}
\item  Gruenwald, P., and Minh, H. N. Evaluation of body and organ
weights in perinatal pathology. I.
    Normal standards derived from autopsies.
{\it Am. J. Clin. Pathol}. 34: 247--53, 1960.
\item Kalter, H. How should times during pregnancy be called in teratology?
{\it Teratology} 1: 231--34, 1968.
\item Kerr, G. R., Kennan, A. L., Waisman, H. A., and Allen, J. R.
Growth and development of the
fetal rhesus monkey. I. Physical growth. {\it Growth} 33: 201--13, 1969.
\item Long, J. A., and Burlingame, M. L. 
The development of the external form of the rat with some
observations on the origin of the extraembryonic coelom and fetal membranes.
{\it Univ. Calif.  Mem. Zool}. 43: 143--83, 1938.
\item Nakatsuka, T., Horimoto, M., Ito, M., Matsuhara, Y.,
  Akaike, M., and F. Ariyuki.  Japan Pharmaceutical
Manufacturers Association (JPMA) survey on background 
control data of development and reproductive toxicity studies in
rats, rabbits and mice. {\it Cong. Anom.} 37:47--138, 1997.
\item Potter, E. L. {\it Pathology of the Fetus and Infant}. Chicago:
Year Book Medical Publishers, 1961.
P. 14.
\item Shepard, T.~H. Growth and development of the human embryo and fetus.
In Gardner, L.~I.
(ed.), {\it Endocrine and Genetic Diseases of Childhood}, 2nd ed.
Philadelphia: W.~B. Saunders, 1975. Pp. 1--6.
\item Shepard, T.~H., Barr, M., Fellingham, G.~W.,
Fujinaga, M., Fitzsimmons, J., and Fantel, A.~G.
Organ weight standards for human fetuses.
{\it Pediat. Pathol.} 4: 513--22, 1988.
\item Tanimura, T., Nelson, T., Hollingsworth, R.~R., and Shepard, T.~H.
Weight standards for organs
from early human fetuses. {\it Anat. Rec}. 171: 227--36, 1971.
\end{references}

\enlargethispage*{1pc}
\vspace*{-1pc}
\begin{numreferences}
\item Boyer, C.~C. Chronology of the development for the golden
hamster. {\it J. Morphol}. 92: 1--37, 1953.

\item Butler, H., and Juurlink, B.~H.~J.
{\it An Atlas for Staging Mammalian and Chick Embryos}. 
Boca Raton, Fla.: CRC Press, 1987. P.~218.

\item Christie, G.~A. Developmental stages in somite and post-somite
rat embryos, based on external appearance and including some
features of the macroscopic development of the oral cavity.
{\it J. Morphol}. 114: 263--86, 1964.
\item Davies, J. {\it Human Developmental Anat\-o\-my}. New York:
Ronald Press, 1963.
\item \refrule, and Hesseldahl, H. Comparative embryology of
mammalian blastocysts. In Blandau,
R.~J. (ed.), {\it Biology of the Blastocyst}. Chicago:
University of Chicago Press, 1971. Pp. 27--48.
\item  Edwards, J.~A. The external development of the rabbit and rat embryo.
In Woollam, D.~H.~M.
(ed.), {\it Advances in Teratology}, vol. 3. New York:
Academic Press, 1968. Pp. 239--63.
\item  Graves, A.~P. Development of the golden hamster
{\it Cricetus auratus} waterhouse, during the
  first nine days. {\it Am. J. Anat}. 77: 219--51, 1949.
\item  Gruenwald, P., and Minh, H.~N. Evaluation of body and organ
weights in perinatal pathology. I.
    Normal standards derived from autopsies.
{\it Am. J. Clin. Pathol}. 34: 247--53, 1960.
\item  Gruneberg, H. The development of some external features in
mouse embryos. {\it J. Hered}. 34:
    88--92, 1943.
\item Hamburger, V., and Hamilton, H.~L.~A series of normal stages in
the development of the chick
embryo. {\it J. Morphol}. 88: 49-92, 1951.
\item Heuser, C.~H., and Corner, G.~W. Development horizons in embryos,
description of age groups
XIX, XX, XXI, XXII, and XXIII, being the fifth issue of a survey of 
the Carnegic Collection. {\it Contrib. Embryol}. 34: 165--96, 1951.
\item Heuser, C.~H., and Streeter, G.~L. Development of the macaque embryo.
{\it Contrib. Embryol}. 29:
15--55, 1941.
\item Huber, G.~C. The development of the albino rat,
{\it Mus norvegicus albinus}. {\it J. Morphol}. 26: 246--386, 1915.
\item Iffy, L., Shepard, T.~H., Jakobovits, A., Lemire, R.~J., and Kerner, P.
The rate of growth in young
embryos of Streeter's horizons XIII to XXIII. {\it Acta Anat}.
(Basel) 66: 178--86, 1967.
\item Kalter, H. How should times during pregnancy be called in teratology?
{\it Teratology} 1: 231--34, 1968.
\item Kerr, G.~R., Kennan, A.~L., Waisman, H.~A., and Allen, J.~R.
Growth and development of the
fetal rhesus monkey. I. Physical growth. {\it Growth} 33: 201--13, 1969.
\item Long, J.~A., and Burlingame, M.~L. 
The development of the external form of the rat with some
observations on the origin of the extraembryonic coelom and fetal membranes.
{\it Univ. Calif.  Mem. Zool}. 43: 143--83, 1938.
\item Monie, I.~W. Comparative development of rat, chick and human embryos.
In {\it Teratologic Workshop Manual} (suppl.). Berkeley, Calif.:
Pharmaceutical Manufacturers Associations,
1965. Pp. 146--62.
\item Nicholas, J.~S. Experimental methods and rat embryos.
In Farris, E.~J., and Griffith, J.~Q.
(eds.), {\it The Rat in Laboratory Investigation}. New York:
Hafner Publishing, 1962. Pp. 51--67.
\item Nishimura, H., and Yamamura, H. Comparison between man and some other mammals of
normal and abnormal developmental processes. In Nishimura, H.,
and Miller, J.~R. (eds.),
{\it Methods for Teratological Studies in Experimental Animals and Man}.
Tokyo: Igaku Shoin,
1969. Pp. 223--40.
\item O'Rahilly, R., and Muller, F. {\it Developmental Stages in Human
Embryos}. Washington, D.C.:
Carnegie Institution of Washington, Publication 637, 1987.
\item Otis, E.~M., and Brent, R. Equivalent ages in mouse and human embryos.
{\it Anat. Rec}. 120: 33--63, 1954.
\item Potter, E.~L. {\it Pathology of the Fetus and Infant}. Chicago:
Year Book Medical Publishers, 1961.
P. 14.
\item Scott, J.~P. The embryology of the guinea pig. I. Table of 
normal development. {\it Am. J. Anat}.
60: 397--432, 1937.
\item Shenefelt, R.~E. Morphogenesis of malformations in hamsters
caused by retinoic acid: Relation to dose and stage of treatment.
{\it Teratology} 5: 103--18, 1972.
\item Shepard, T.~H. Growth and development of the human embryo and fetus.
In Gardner, L.~I.
(ed.), {\it Endocrine and Genetic Diseases of Childhood}, 2nd ed.
Philadelphia: W.~B. Saunders, 1975. Pp. 1--6.
\item Shepard, T.~H., Barr, M., Fellingham, G.~W.,
Fujinaga, M., Fitzsimmons, J., and Fantel, A.~G.
Organ weight standards for human fetuses.
{\it Pediat. Pathol}. 4: 513--22, 1988.
\item Streeter, G.~L. Developmental horizons in human embryos,
description of age group XI, 13 to
20 somites, and age group XII, 21 to 29 somites.
{\it Contrib. Embryol}. 30: 211--45, 1942.
\item \refrule. Developmental horizons in human embryos,
description of age group XIII embryos
about 4 or 5 millimeters long, and age group XIV,
period of indentation of lens vesicle. 
{\it Contrib.  Embryol}. 31: 27--63, 1945.
\item \refrule. Developmental horizons in human embryos,
description of age groups XV, XVI, XVII
and XVIII, being the third issue of a survey of the Carnegie Collection.
{\it Contrib. Embryol}. 32: 133--203, 1948.
\item Tanimura, T., Nelson, T., Hollingsworth, R.~R., and Shepard, T.~H.
Weight standards for organs
from early human fetuses. {\it Anat. Rec}. 171: 227--36, 1971.
\item Theiler, K. {\it The House Mouse Development and Normal Stages
from Fertilization to 4 Weeks
of Age}. Berlin: Springer-Verlag, 1972.
\item Witschi, E. Development of the rat. In Altman, P.,
and Dittmer D.~S. (eds.), {\it Growth Including
Reproduction and Morphological Development}. Washington, D.C.:
Federation of American
Societies for Experimental Biology, 1962. Pp. 304--414.
\vspace*{3pc}
\
\end{numreferences}



\clearemptydoublepage
\chapter[Human Teratogens: Proven, Possible, and Unlikely]{Human Teratogens}

\vspace*{-1pc}
{\sffamily \Large       Proven, Possible, and Unlikely (February, 2007)}
\vspace*{2pc}



\label{cfht}
\noindent
Several of our associates have encouraged us to include lists of
agents that are considered to be proven, possible,
or unlikely teratogens in humans. This has been
done with some trepidation, since we have attempted in the textual
material to present
the facts without comment. The substantiation for inclusion of an
agent in one of these
three lists has been drawn from data recorded in the text. The
criteria used for proof of human teratogenicity are listed in
table 1 and represent an amalgamation from writings by Wilson
(1977), Brent (1978), Stein et al.~(1984), Hemminki and Vineis
(1985), and Shepard (1973--2007, 1986).

In recent years, several papers on teratogenic risk have appeared.
Jelovsek et al.~(1989) analyze the relationship between animal tests
and the identification of human teratogens. In another work, Jelovsek
et al.~(1990) weigh the principles used by experts to identify and
classify human teratogens. A paper by Kimmel and Gaylor (1988)
discusses qualitative and quantitative evaluations of risk and how
``acceptable'' exposure levels are determined. A scheme for ranking
chemicals by their potential risk is detailed in Wang and
Schwetz (1987).

The Teratology Society Public Affairs Committee (1994)
has urged the FDA to abandon its teratogenic rating system,
because it may produce undue anxiety in the physician and pregnant 
woman and contribute to unnecessary termination of wanted pregnancies.

\begin{references}

\item Brent, R. L. Editor's note. {\it Teratology} 17: 183, 1978.
\item Hemminki, K., and Vineis, P. Extrapolation of the evidence on
teratogenicity of chemicals between
humans and experimental animals: Chemicals other than drugs.
{\it Teratog., Carcinog., and Mutagen.} 5: 251--318, 1985.
\item Jelovsek, F. R., Manison, D. R., and Chen, J. J. 
Prediction of risk for human developmental toxicity:
How important are animal studies for hazard identification?
{\it Obstet. Gynecol}. 74: 624--36, 1989.
\item Jelovsek, F. R., Manison, D. R., and Young, J. F.
Eliciting principles of hazard identification from
experts. {\it Teratology} 42: 521--33, 1990.
\item Kimmel, C. A., and Gaylor, D. W.
Issues in qualitative and quantitative risk analysis for developmental toxicity.
{\it Risk Anal.} 8: 15--20, 1988.
\item Shepard, T. H. {\it Catalog of Teratogenic Agents}, 1st--10th eds.
Baltimore: Johns Hopkins University
Press, 1973--2001.
\item Shepard, T. H. Human teratogenicity. In Barness, L. (ed.),
{\it Advances in Pediatrics}, vol. 33. Chicago:
Year Book Medical Publishers, 1986. Pp. 225--68.
\item Shepard, T. H., and Lemire, R.~J.  {\it Catalog of Teratogenic Agents}, 11th--12th eds. 
Baltimore: Johns Hopkins University Press, 2004--2007.
\item Stein, Z., Kline, J., and Kharrazi, M. What is a teratogen?
Epidemiologic criteria. In Kalter, H. (ed.),
{\it Issues and Reviews in Teratology}, vol. 2. New York:
Plenum Press, 1984. Pp. 23--66.
\item Teratology  Society Public Affairs Committee.
FDA classification of drugs for teratogenic risk.
{\it Teratology} 49: 446--47, 1994.
\item Wang, G. M., and Schwetz, B. A. An evaluation system for ranking
chemicals with teratogenic potential. {\it Teratog.,
Carcinog., and Mutagen.} 7: 133--39, 1987.
\item Wilson, J. G. Embryotoxicity of drugs in man. In Wilson, J. G.,
and Fraser, F. C. (eds.), {\it Handbook of
Teratology}, vol. 1. New York: Plenum Press, 1977. Pp. 309--55.
\end{references}



\def\Hrule{\vskip3pt\hrule\vskip3pt}
\begin{table}[!b]
\mediumlarge
\noindent {\sffamily TABLE 1. \textsl{Amalgamation of Criteria for Proof
of Human Teratogenicity}}

\Hrule
\begin{enumerate}

\item Proven exposure to agent at critical time(s) in prenatal development (prescriptions, physician's
records, dates).

\item Consistent findings by two or more high quality epidemiologic studies:
\begin{enumerate}
\itemindent 4pt
\item Control of confounding factors;
\item Sufficient numbers;
\item Exclusion of positive and negative bias factors;
\item Prospective studies, if possible; and
\item Relative risk of six or more (?).
\end{enumerate}

\item Careful delineation of the clinical cases. A specific defect or syndrome, if present, is very
helpful.
\item Rare environmental exposure associated with rare defect. Probably three or more cases
(examples: oral anticoagulants and nasal hypoplasia,
methimazole and scalp defects (?), and heart block and
maternal rheumatic disease).
\item Teratogenicity in experimental animals important but not essential.
\item The association should make biologic sense.
\item Proof in an experimental system that the agent acts in an unaltered state. Important information
for prevention.
\end{enumerate}

\Hrule
{\noindent\normalsize
Note: Items 1, 2, and 3 or 1, 3, and 4 are essential criteria. Items 5, 6, and 7 are helpful but not essential.
\par}
\end{table}


\def\tblh#1{\textit{#1}\\}
\def\tbll#1{\hspace*{1pc}#1\\}


\begin{table}[p]
\mediumlarge
\everypar{\hangindent1pc\hangafter1}
\noindent {\sffamily TABLE 2. \textsl{Teratogenic Agents in Human Beings}}

\Hrule
\begin{tabular*}{\halfcolsiz}{@{}p{\halfcolsiz}@{}}
\noalign{\vspace*{0pt}}
\tblh{Radiation}
\tbll{Atomic weapons}
\tbll{Radioiodine}
\tbll{Therapeutic}
\noalign{\vspace*{6pt}}
\tblh{Infections}
\tbll{Cytomegalovirus (CMV)}
\tbll{Herpes simplex virus 1 and 2}
\tbll{Lymphocytic choriomeningitis}
\tbll{Parvovirus B-19}
\tbll{\ \ \ (Erythema infectiosum)}
\tbll{Rubella virus}
\tbll{Syphilis}
\tbll{Toxoplasmosis}
\tbll{Varicella virus}
\tbll{Venezuelan equine encephalitis virus}
\noalign{\vspace*{6pt}}
\tblh{Maternal and Metabolic Imbalance}
\tbll{Alcoholism}
\tbll{Amniocentesis, early}
\tbll{Chorionic villus sampling}
\tbll{\ \ \  (before day 60)*}
\tbll{Cretinism, endemic}
\tbll{Diabetes}
\tbll{Folic acid deficiency}
\tbll{Hyperthermia}
\tbll{Myasthenia gravis}
\tbll{Phenylketonuria}
\tbll{Rheumatic disease and congenital}
\tbll{\ \ \  heart block}
\tbll{Sjogren's syndrome}
\tbll{Virilizing tumors and congenital}
\tbll{\ \ \  adrenal hyperplasia}
\noalign{\vspace*{24pt}}
\end{tabular*}
\begin{tabular*}{\halfcolsiz}{@{}p{\halfcolsiz}@{}}
\noalign{\vspace{6pt}}
\tblh{Drugs and Environmental Chemicals}
\tbll{Aminopterin and methylaminopterin}
\tbll{Androgenic hormones}
\tbll{Busulfan}
\tbll{Captopril (renal failure)}
\tbll{Carbamazepine*}
\tbll{Chlorobiphenyls}
\tbll{Cigarette smoking}
\tbll{Cocaine}
\tbll{Corticosteroids*}
\tbll{Coumarin anticoagulants}
\tbll{Cyclophosphamide}
\tbll{Cytarabine}
\tbll{Danazole}
\tbll{Diethylstilbestrol}
\tbll{Diphenylhydantoin}
\tbll{Enalapril (renal failure)}
\tbll{Etretinate}
\tbll{Fluconazole, high dose}
\tbll{Iodides and goiter}
\tbll{Lithium*}
\tbll{Mercury, organic}
\tbll{\hbox{Methimazole and scalp defects and choanal atresia*}}
\tbll{\hbox{Methylene blue via intra-amniotic injection}}
\tbll{Misoprostol*}
\tbll{Nitrogen mustard}
\tbll{Penicillamine}
\tbll{Phenobarbitol*}
\tbll{\hbox{1 3-cis-Retinoic acid (Isotretinoin and Accutane)}}
\tbll{Sartans}
\tbll{Tetracyclines}
\tbll{Thalidomide}
\tbll{Toluene abuse}
\tbll{Trimethadione}
\tbll{Valproic acid}
\end{tabular*}
\Hrule
* Denotes agents that produce less than 10 defects in 1,000 exposures.
 \end{table}

% \bigskip

 \begin{table}[p]
 \mediumlarge

\everypar{\hangindent1pc\hangafter1}
\noindent {\sffamily TABLE 3. \textsl{Possible and Unlikely Teratogens}}

\Hrule
\begin{tabular*}{\halfcolsiz}{@{}p{\halfcolsiz}@{}}
% \noalign{\vspace{-16pt}}
\tblh{Possible Teratogens}
 \tbll{Binge drinking}
 % \tbll{Carbamazepine}
 \tbll{Colchicine}
 \tbll{Disulfiram}
 \tbll{Ergotamine}
 \tbll{Glucocorticoids}
 \tbll{High Vitamin A}
 \tbll{Lamotrigine}
 \tbll{Lead}
 \tbll{Paroxetine}
 \tbll{Primidone}
 \tbll{Pseudoepinephrine}
 \tbll{Quinine, suicidal doses}
 \tbll{Streptomycin}
 \tbll{Zidivudine (AZT)}
 \tbll{Zinc deficiency}
\noalign{\vspace*{24pt}}
\end{tabular*}
\begin{tabular*}{\halfcolsiz}{@{}p{\halfcolsiz}@{}}
% \noalign{\vspace{6pt}}
\tblh{Unlikely Teratogens}
\tbll{Agent Orange}
\tbll{Anesthetics}
\tbll{Aspartame}
\tbll{Aspirin (but aspirin in the 2nd half of}
\tbll{\ \ \ pregnancy may increase cerebral}
\tbll{\ \ \ hemorrhage during delivery)}
\tbll{Bendectin (antinauseants)}
\tbll{Illicit drugs (marijuana, LSD)}
\tbll{Metronidazole (Flagyl)}
\tbll{Oral contraceptives}
\tbll{Progesterone (Hydroxyprogesterone}
\tbll{\ \ \  and Medroxyprogesterone)}
\tbll{Rubella vaccine}
\tbll{Spermicides}
\tbll{Ultrasound}
\tbll{Video display terminals and}
\tbll{\ \ \  electromagnetic waves}
\end{tabular*}
\Hrule
\end{table}

\clearemptydoublepage
\chapter{Use of the Catalog}


The text of the catalog consists of 
entries for the teratogenic agents, in alphabetical order.
Chemical compounds are alphabetized without regard
for single preceding letters, numbers, or Greek symbols. For
instance, I-beta-D-arabinofuranosyl-5-fluorocytosine would be listed
in the A's.  The chemical name as listed in the {\it Merck Index} has
been used most often as the main entry, with trade names as synonyms.
Trade names are followed by \tmm. Chemical abstract numbers (CAS)
are supplied when possible.  Following the
catalog are the author and agent indexes.
If there are various names for a single agent, the
reader can locate the text entry more rapidly by consulting the
agent index. For example, the listing for irradiation in the agent
index refers
the reader to the text entry under radiation.  In both indexes, the
numbers supplied refer to {\it entries}, not pages.  
Note that page numbers for the book appear
at the top of each page, near the inside margin; entry numbers are
in boldface, near the outside margin.


% new stuff for 10th edition

\clearemptydoublepage
\chapter{Developmental Gene Mutations as Teratogenic Agents}

\label{gene}
Why should gene mutations be in this book on teratology?  There are
more than 300 genes listed in this edition, with an emphasis on those that
produce a congenital defect found in the newborn period.  F. Clarke
Fraser long ago impressed Dr.~Shepard that a mutated gene should be
considered as yet another teratogenic agent.  Since teratologists are
basically embryologists and have the knowledge and tools to study
critical events happening in the embryo/fetus, this new information
offers exciting challenges.  The genes, or their abbreviations, are
listed alphabetically in the text and index.  The syndromes are also
cross-referenced in the index to the gene.  In some cases the
syndrome is the main entry, with the gene added as a synonym.  As of
this writing (Feb. 21, 2007) about 25 percent of birth defect syndromes have
been assigned to a specific gene or deletion.

A small beginning is being made to study the interaction of genes
with other agents.  Three examples are the genotypes of folic acid
metabolism with mostly unknown initiators of neural tube defects
(Shepard and Lemire, 2007),
the role of the HOXA genes in autism (Ingram et al.,
2000), and the contribution of transforming growth factor alpha to
cleft palate in smokers (Werler, 1997).

The main source of information on the altered genes and the
malformation syndromes they produce has been Victor McKusick's Online
Mendelian Inheritance in Man (OMIM), 
www.ncbi.nlm.nih.gov/omim (for PWS [MIM176270]).  Boyadjiev
and Jabs (2000) have listed in table form 162 developmental genes
along with the phenotype that the mutation produces.  They divide
them into organ categories and include the mouse models.  A
convenient entry into the mouse models is the Human-Mouse Homology
Map,
www.ncbi.nih.gov/homology.
Other recent reviews which have been useful are on the central
nervous system (Sarnat and Menkes, 2000), homeobox genes (Mark et
al., 1997), limb development (Bamshad et al., 1999), and kidney
(Piscione and Rosenblum, 1999).



\begin{references}



\item Bamshad, M., Watkins, W.~S., Dixon, M.~E., Le, T., Roeder, A.~D.,
Kramer, B. E., Cavey, J. C., and Jorde, L. B.
Reconstructing the history of human limb development:
Lessons from birth defects. {\it Pediat. Res.} 45: 291--99, 1999.

\item Boyadjiev, S. A., and Jabs, E. W.
Online Mendelian Inheritance in Man (OMIM) as a knowledge base
for human developmental disorders.  {\it Clin. Genet.}  57: 253--66, 2000.

\item Ingram, J. L., Stodgell, C.~J., Hyman, S.~L., Figlewicz, D.~A.,
Weitkamp, L.~R., and Rodier, P.~M.
Discovery of allelic variants of HOXAI and HOXBI:
Genetic susceptibility to autism spectrum disorders.
 {\it Teratology} 62: 393--405, 2000.

\item Mark, M., Filippo, M. R., and Chambon, P.
Homeobox genes in embryogenesis and pathogenesis {\it Pediat. Res.} 
42: 421--29, 1997.

\item Piscione, T. D., and Rosenblum, N.~D. 
The malformed kidney: Disruption of glomerular and tubular development.
 {\it Clin. Genet.}  56: 341--56, 1999.

\item Sarnat, H. B., and Menkes, J.~H.  How to construct a neural tube.
{\it J. Child Neural.}  15: 110--24, 2000.

\item Shepard, T. H., and Lemire, R.~J.  {\it Catalog of Teratogenic Agents}. 
12th ed. Baltimore: Johns Hopkins University Press, 2007.

\item Werler, N. M.  Teratogen update: Smoking and reproductive outcomes. 
{\it Teratology} 55: 382--88, 1997.

\end{references}

